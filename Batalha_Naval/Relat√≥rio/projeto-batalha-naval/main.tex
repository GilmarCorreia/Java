\documentclass[]{article}
\usepackage[utf8]{inputenc}
\usepackage[T1]{fontenc}
\usepackage{graphicx}
\usepackage[portuguese]{babel}

%opening
\title{<<Batalha Naval>>}
\author{11019915-Allan Calixto Severo da Silva,\\
11035315-Ana Laura Belotto Claudio,\\
11014515-Gilmar Correia Jeronimo,\\
11110415-Nickolas Lopes Pimenta}
\date{\today}

\begin{document}

\maketitle

\centerline{\textbf{Turma: A1 - Matutino}}

\section{Introdução}
O projeto é o jogo de estratégia chamado \textit{BattleShip} (Batalha Naval) e foi implementada utilizando conceitos de programação orientada à objetos, na linguagem Java. No projeto foram implementados duas opções de jogo, uma \textit{single-player}, cujo o jogandor joga conta o computador, e uma com a opção de \textit{versus  mode}, ou seja, o \textit{player} 1 contra o \textit{player} 2. Nesse jogo ganha quem destruir primeiro todas as embarcações do adversário, tentanto advinhar onde elas estão escondidas no campo, com uma tentativa (tiro) por vez.

Caso o jogador acerte um barco, ele tem a chance de continuar atirando para adivinhar a posição do barco, caso erre o tiro a vez passa para o oponente. Para esse projeto foi desenvolvido uma interface gráfica para comunicar os jogadores de maneira mais interativa, no qual é possível utilizar o mouse para setar posições, mudar orientações do barco e atirar. Para isso foi feito uso do JavaFX, garantindo a jogabilidade. 

\section{Descrição das classes}
\begin{itemize}
	\item Classe Principal: inicia o jogo, ou seja,  inicia a interface gráfica e as sub-classses que fazem o jogo funcionar.
	
    \item Classe Tela: define qual é a cena que vai ser mostrada no JavaFx, além do título, imagem e fonte. É uma classe abstrata. É a classe-mãe de outras classes, cada uma define uma cena a ser mostrada, temos as subclasses Regras, TelaMain, Escolher, GameScene e ButtonClass.
    \item Classe Regras (subclasse de Tela): é uma sub-classe da classe Tela. Responsável pelas regras do jogo.
    \item Classe ButtonClass (subclasse de Tela): tem como objetivo retornar um botão. É enviado uma cena e o botão retorna qual é a próxima cena, o texto e a fonte.
    \item Classe GameScene (subclasse de Tela): define a mecânica do jogo, qual o evento a ser realizado pelo Player, por exemplo quem começa jogando, qual a posição do barco, entre outras funções.
    \item Classe Escolher (subclasse de Tela): mostra na tela do computador os botões dos modos de jogo, a opção de voltar, etc.
    \item Classe TelaMain (subclasse de Tela): responsável pela tela inicial do jogo.
    \item Classe GamePlayerFunctions: é uma interface que define o que cada Player deve ter, como o método atirar, verificar o tiro que levou, validar posição para uma dada coordenada do barco. É implementada por Computer e Player.
    \item Classe Computer (subclasse de GamePlayerResources e implementa GamePlayerFunctions): realiza a mecânica de jogo do Player Computador. Ela cria uma matriz de probabilidade e por meio dela são definido onde serão os tiros. 
    \item Classe Player(subclasse de GamePlayerResources e implementa GamePlayerFunctions): valida as coordenadas de tudo que o jogador clicar com o mouse ou com o teclado.
	\item Classe GamePlayerResources (subclasse de VBox e implementa GamePlayerFunctions): É a uma das classes fundamentais do jogo, pois é ela que vai definir toda a comunicação da interface visual com as embarcações. Essa classe extende uma "V.box" que vai definir a imagem de como o player vai aparecer, ela também cria as matrizes, verifica se tem embacação na célula ou não, verifica se eu já a célula já foi jogada, setar as embacações nas posições, identifica qual embacação foi atingida se o palpite do jogador for certo, e verifica a vida dos players.   
    \item Classe Shot: é uma classe do tipo Enun, que enumera uma lista de tiros possíveis no jogo, que seriam: afundou, atingiu e errou.
    \item Classe Coordenate: Essa classe vai setar uma coordenada x,y e retorna-lá.
	\item Classe Célula: Essa classe vai criar um dos quadrados da matriz e verifica se existe ou não alguma embacação naquela posição, caso exista ela vai colorir para o jogador saber que ele acertou alguma coisa, caso contrário ela modifica para cor de célula já jogada.
	\item Classe Embarcacao: é a classe-mãe de todas as embarcações, Submarino, PortaAviao, NavioTanque e Contratorpedeiro. É responsável pelo tamanho do barco, a posição e se está na vertical ou não. É uma classe abstrata. 	\item Classe Contratorpedeiro (subclasse de Embarcacao): Instancia a embarcação Contratorpedeiro, com os métodos da classe-mãe.
	\item Classe NavioTanque (subclasse de Embarcações): Instancia a embarcação NavioTanque, com os métodos da classe-mãe.
	\item Classe PortaAviao (subclasse de Embarcações): Instancia a embarcação PortaAviao, com os métodos da classe-mãe.
	\item Classe Submarino (subclasse de Embarcações): Instancia a embarcação Submarino, com os métodos da classe-mãe.
	\item Classe Exception: Responsável por lançar uma exceção caso o jogador tente jogar na mesma que ele já tenha jogado.
\end{itemize}

\section{Conceitos de orientação a objetos aplicados}
Os seguintes conceitos foram utilizados no projeto:

\begin{itemize}
	\item Construtores
	\item Herança
	\item Interface
	\item Encapsulamento
    \item Classe Abstrata
	\item Tratamento de exceções
    \item Padrão de Software
\end{itemize}

Todos esses conceitos foram implementados de forma adequada utilizando os conceitos de POO, seja construtores para quando uma classe for chamada, herança para ter uma super classe com atributos e métodos com sub-classes que possam ter acesso a esses métodos, encapsulamento com os atributos e métodos com as restrições de acesso adequadas, tratamento de exceções para caso se jogador tentar fazer uma jogada ilegal e também foi utilizado uma interface para padronizar as funções dos jogadores, tanto player quanto computer.

Foi utilizado um padrão de software para controlar a mecânica dos jogadores, o  \textit{Strategy}. Esse padrão é geralmente aplicado quando muitas classes relacionam-se, no caso do projeto temos várias classes que atuam de forma diferente, mas se relacionam para que o jogo funcione. No projeto o \textit{strategy} é utilizado para chamar hora um jogador humano, hora uma máquina. Com a interface criada é possível sobreescrever os métodos e setar para cada uma das classes implementadas os métodos atira, verifica posição e retorna tiro, ou seja, método que define uma coordenada para ser atirada, método que auxilia no posicionamento dos barcos e um que retorna qual tipo de tiro foi dado.

\section{Participação de cada integrante do grupo}

Ana Laura: responsável pela classe das embarcações, pela inteligência do \textit{player} computador, ajustes no códigos das classes e revisão do relatório e criaçãoo do diagrama UML.
\\
Gilmar Correia: responsável pela interface gráfica, integração das classes como um todo para o funcionamento do projeto e revisão do relatório.
\\
Nickolas Pimenta: responsável pela classe do jogador e pelo desenvolvimento do relatório.
\\
Allan Calixto: Responsável no suporte do trabalho em geral.

\end{document}
